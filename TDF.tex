\documentclass[conference, a4paper,10pt,twocolumn]{IEEEtran}

\usepackage{graphics} % for pdf, bitmapped graphics files
\usepackage{epsfig} % for postscript graphics files
\usepackage{mathptmx} % assumes new font selection scheme installed
\usepackage{times} % assumes new font selection scheme installed
\usepackage{amsmath} % assumes amsmath package installed
\usepackage{amssymb}  % assumes amsmath package installed
\usepackage{psfrag}
\usepackage{subfigure}
\usepackage{cite}
\usepackage{amsthm}

% correct bad hyphenation here
\hyphenation{op-tical net-works semi-conduc-tor IEEEtran}

\begin{document}

\title{A simple approach to WSN monitoring}

% avoiding spaces at the end of the author lines is not a problem with
% conference papers because we don't use \thanks or \IEEEmembership

% use only for invited papers
%\specialpapernotice{(Invited Paper) }

% make the title area
\maketitle
%\footnote{}
\begin{abstract}

We propose and explore the use of a very simple concept including data encoding
that can be used to enable urgent applications for Wireless Sensor Networks (WSN), 
monitoring and data collection. We also show how data can be distributed, shared 
stored and plotted on different platforms. The overall concept follows a straight 
forward unix-style format. Data is in ASCII and labeled with simple tags. 
It's designed to be easy to debug with low complexity and protocol support. We found 
this an alternative for monitoring applications and are using the format and ecosystem 
in long term field tests and projects.


\end{abstract}

\begin{IEEEkeywords} 
Internet of Things, Wireless sensor networks, WSN, Contiki, RIME
 \end{IEEEkeywords}


% no keywords

% For peer review papers, you can put extra information on the cover
% page as needed:
% \begin{center} \bfseries EDICS Category: 3-BBND \end{center}
%
% for peerreview papers, inserts a page break and creates the second title.
% Will be ignored for other modes.
% \IEEEpeerreviewmaketitle

\section{Introduction}
\label{sec:intro}
 

The motivation for this work is the need for innovative solutions to use the advances 
WSN and IoT research. While some areas is still under development the need for the
simplest solution to address urgent application was discovered. Also there were 
strong needs for simple installation, including setup, minimal configuration and 
simple debugging.

Microcontrollers are available, even microcontrollers with builtin radio tranceivers 
as AtMega128RFA1 ~\cite{ATMEGA}. Implementing a IEEE 802.15.4 ~\cite{802154} aslo operating 
systems like Contiki is availbe ~\cite{CONTIKI} and mature. Contiki includes RIME ~\cite{CONTIKI} 
which is a very small and simple protocol for WSN commonication.


In section ~\ref{sec:needs} , we discuss a more detailed needs and requirement analysis for the use cases
In section ~\ref{sec:implementation}, we present details about current implementation  
In section ~\ref{sec:experince}, we present an installation and actual usage
Finally, in section ~\ref{sec:conclusion}, we present our conclusions and suggestions for further studies.

\section{Needs and requirements}
%%\label{sec:Needs and requirements}
\label{sec:needs}


Different projects has motivated us for this work. One of the recent use cases is wireless sensor 
networks for meteorological and environmental monitoring ~\cite{WIMEA}. 
Such networks are often deployed in remote areas with little or no power supply~\cite{UBIQUI}.  
Our low power hardware can be placed in deep sleep between measurements and transmissions.  The key 
components in our design include  Contiki-OS~\cite{CONTIKI} running on Atmel ATMega128RF~\cite{ATMEGA} 
integrating an MCU, an IEEE 802.15.4-compatible~\cite{802154} radio transceiver and an 8-channel 10-bit 
AD-converter. This component has been used to design a WSN mote with some additional components such 
as an EUI64 address chip, voltage converters, a temperature sensor, etc. When in deep sleep, the MCU 
itself consumes ~1 $\mu$A and the entire mote ~10-15 $\mu$A. When transmitting, the mote consumes 
~20mA. At 3V, this means ~60mW. 


\section{Implementation}
\label{sec:implementation}

\subsection{Light-weight protocol ideas}

IoT devices and sensors has three major mapping to the Internet. The 
or node cen exposed via

\begin{itemize}

\item Native IP address. Ether IPv4 or IPv6. 6LowPAN constitutes IPv6


\item Via a IP gateway. Gateway can of course have IPv4 or IPv6. Protocol
can be any native IP protocols or IP protocols specially designed 
for constrained applications like CoAP or MQTT etc. The gateway 
translates between two standardized internet protocol stacks. 


\item  Via IP gateway but in this case a non IP address are used on
the nodes examles Contki's RIME, Zigbee . The gateway translates 
between two protocol stacks. 

\end{itemize}

I our implementations we've he used. RIME for the WSN-nodes. I should
be pointed out that our encoded data does not require this.

RIME is a lightweight protocol with a very low overhead. Header is 
16 bit's. The current implementation uses a RIME broadcast network
the 16 bit source address is composed by the unique EUI64 address
chip on each node.




\subsection{Network topology}
Here

\subsection{Report and sink nodes}
Here

\subsection{Power needs}
Here

\subsection{Data format}
Here

Stateless vs stateful monitoring can be discussed. 


\subsection{Gateway and proxy}
Here we discuss sensd ~\cite{sensd}

\subsection{Plot and utilities}
Here

\subsection{App support}
Here we discuss Read Sensors App  ~\cite{read-sensors}

\subsection{Data repository and storage}
Here

\section{Experiences and Installations}
\label{sec:experince}
Here we discuss WIMEA ~\cite{WIMEA}  ~\cite{WIMEAREPORT}

\section{Conclusions and further work}
\label{sec:conclusion}

We see research directions to follow:

\begin{itemize}
\item Access control of nodes and data. 

\item Merge of data flows 

\item If possible an more compact data representation and alinment to CoAP, MQTT etc

\end{itemize}

Issues for further study include: 

\begin{itemize}
\item Scalability 
\itemHow size of WSN network.

\item Security
\itemHow data integrity and security

\end{itemize} 

The authors are currently involved in field studies including these issues

\begin{thebibliography}{1}

\bibitem{UBIQUI} Amos Nungu, Robert Olsson, Bj\"{o}rn Pehrson. \emph{Implementation of Inclusive Ubiquitous Access}. 
Journal of Wireless Personal Communication, 2012

\bibitem{802154}  \emph{IEEE 802.15.4}. 
[Online]. Available: http://www.ieee802.org/15/pub/TG4.html. [Accessed: 21-February-2012]

\bibitem{CONTIKI}  \emph{The Contiki Os.}. 
[Online]. Available: http://www.contiki-os.org/. [Accessed: 29-February-2012].

\bibitem{RIME} Adam Dunkels. \emph{Rime - a lightweight layered communication stack for sensor networks}.   In Proceedings of the European Conference on Wireless Sensor Networks (EWSN), Poster/Demo session, Delft, The Netherlands, January 2007.

\bibitem{ATMEGA} Atmel Corporation. \emph{ATmega128RFA1}. 
[Online]. Available: http://www.atmel.com/devices/ATMEGA128RFA1.aspx. [Accessed: 21-February-2012].

\bibitem{RSS2} Radio Sensors mote. \emph{RSS2}. 
[Online]. Available: http://radio-sensors.com/ [Accessed: 16-June-2015]

\bibitem{LICCAP} Robert Olsson, Bj\"{o}rn Pehrson. \emph{Powering devices using ultra-capacitor batteries}
[Publishing Pending]

\bibitem{UBIQUISTATUS} [Amos Nungu, Robert Olsson, Bj\"{o}rn Pehrson, Jiawei Kang, Daniel Kifetew, Alisher Rustamov]. \emph{Inclusive Ubiquitous Access - A Status Report}. 
Africom, Younde, Nov 2012

\bibitem{SBN} SBN. \emph{IL2213 WSN-Projects Fall 2011.}. 
[Online]. Available: http://www.tslab.ssvl.kth.se/csd/files/wsn/index.html. [Accessed: 21-February-2012].

\bibitem{sensd}  \emph{sensd gateway}. 
[Online]. Available: https://github.com/herjulf/sensd [Accessed: 16-June-2015]

\bibitem{read-sensors}  \emph{Read-Sensors Android app}. 
[Online]. Available: https://github.com/herjulf/Read-Sensors [Accessed: 16-June-2015]

\bibitem{WIMEA}  \emph{WIMEA-ICT project.}. 
[Online]. Available: http://wimea-ict.gfi.uib.no/. [Accessed: 21-March-2015].

\bibitem{WIMEAREPORT}  \emph{WIMEA-ICT Progress Report}. 
[Online]. Available: http://wimea-ict.gfi.uib.no/. [Accessed: 21-March-2015].

\bibitem{ODROID} \emph{Odroid U3 and C1}
[Online]. Available: http://www.hardkernel.com [Accessed 5-April-2015]
\end{thebibliography}
\end{document}

